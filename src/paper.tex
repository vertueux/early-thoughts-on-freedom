\documentclass[]{cls/tools}

\usepackage[french]{babel}
\usepackage{url}

\begin{document}

\title{Réflexion sur la liberté}

\author{VERTUEUX}
\date{October 31, 2022}

\maketitle

\section*{Introduction}
La liberté dans notre société actuelle est une source à débat parmi tous les individus et politiciens de ce monde. Que cela soit en 
argumentant du droit des personnes fortunés, des personnes détenant des difficultés financières, de l'inégalité globale et surtout 
des différences théoriquement changeables. 

Cependant, la question et la doctrine que soulèvera cet article est sur la notion de cette dernière. En effet, remettre en question 
ce qu'est concrètement et véritablement la liberté, semblant notamment être reformulée perpétuellement.

\section*{I - La société}
Si nous nous basons sur une définition simple, universelle de ce qu'est la liberté, nous 
pouvons la résumer à l'absence de contrainte, la non-soumission à la servitude et la capacité 
de la conscience d'agir et de penser selon la volonté de l'individu.\newline

La société française, si nous ne considérons pas de même l'ensemble des diverses sociétés démocratiques au sein de cette planète,
divise libertés et droits fondamentaux en diverses parties perçues par la loi selon des principes simples
et concrets. Hors, tout cela qui implique déjà une modification et redéfinition de la liberté. 

En effet, la liberté est une notion philosophique abstraite, si ce n'est subjective. Il est assez complexe,
si ce n'est pas indémontrablement impossible, de vouloir l'inscrire dans un regroupement d'idées diverses 
cherchant éperdument l'accord de tous. Comment serait-il possible de trouver l'objectivité dans un monde de nature subjective ?

Réellement, une telle reformulation et redéfinition concrète de la liberté implique une limitation sur son sens originel.
Une comparaison allégorique pouvant servir de support afin d'illustrer la limitation sociétale imposée par le gouvernement pourrait être la suivante:\newline

Imaginez plusieurs primates enfermé dans une cage partiellement ouverte, dans un espace entièrement rempli d'autres cages. Ces singes sont nés et ont vécu 
leur entière vie au sein de cette cage. Un homme offre à ces derniers une 
certaine alimentation et liberalité relativement aux actions demandées qui ont été réalisées. Ces derniers sont libres d'effectuer ce 
qu'ils souhaitent dans leur cage, tant que cela ne soit pas susceptible de la démanteler. S'ils sont insatisfaits, ils sont libres de sortir 
de la cage et d'inéluctablement passer dans une autre. Enfin, ils ne chercheront même pas à s'émanciper et chercher ce qu'il y au delà de ce monde 
rempli de cages, car il s'agit de l'unique chose qu'ils ont connus. Ils ont donc la liberté d'opérer de manière restrictive dans une cage ainsi que de d'être déplacé dans une autre. Tel est véritablement cette liberté 
sociétale que chaque être humain reçoit à la naissance.\newline

C'est donc par un raisonnement similaire à l'allégorie de la caverne de Platon que l'on peut décrire la liberté sociétale.

Quel est donc une société démocratique qui se donne des valeurs et objectifs à atteindre, en modifiant leur sens ? 
Dans un tel cas, il serait par exemple commode d'offrir le bonheur au malheureux, quand le bonheur a été redéfini comme tout simplement
de l'argent.\newline

\end{document}